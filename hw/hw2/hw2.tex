\documentclass[UTF8]{article}
\usepackage{graphicx}
\usepackage{subfigure}
\usepackage{amsmath}
\usepackage{makecell}
\usepackage[utf8]{inputenc}
\usepackage[space]{ctex} %中文包
\usepackage{listings} %放代码
\usepackage{xcolor} %代码着色宏包
\usepackage{CJK} %显示中文宏包
\usepackage{float}
\usepackage{diagbox}
\usepackage{bm}
\usepackage{ulem} 
\usepackage{amssymb}
\usepackage{soul}
\usepackage{color}
\usepackage{geometry}
\usepackage{fancybox} %花里胡哨的盒子
\usepackage{xhfill} %填充包, 可画分割线 https://www.latexstudio.net/archives/8245
\usepackage{multicol} %多栏包
\usepackage{enumerate} %可以方便地自定义枚举标题
\usepackage{enumitem}
\usepackage{multirow} %表格中多行单元格合并
\usepackage{wasysym} %可以使用wasysym里的一堆奇奇怪怪的符号
\usepackage{hyperref} % url
%%%%%%%%%%%%%%%伪代码%%%%%%%%%%%%%%%
\usepackage{amsmath}
\usepackage{algorithm}
\usepackage{algorithmicx}
\usepackage[noend]{algpseudocode}
%%%%%%%%%%%%%%%画图包%%%%%%%%%%%%%%%
\usepackage{tikz}
\usepackage{pgfplots} % http://pgfplots.sourceforge.net/gallery.html
\usetikzlibrary{pgfplots.patchplots} % 拟合支持
\usetikzlibrary{arrows,shapes,automata,petri,positioning,calc} % 状态图支持
\usetikzlibrary{arrows.meta} % 箭头
\usetikzlibrary{shadows} % 阴影支持
\usepackage{forest} % 画树

\geometry{left = 1.5cm, right = 1.5cm, top=1.5cm, bottom=2cm}

\definecolor{mygreen}{rgb}{0,0.6,0}
\definecolor{mygray}{rgb}{0.5,0.5,0.5}
\definecolor{mymauve}{rgb}{0.58,0,0.82}
\lstset{
	backgroundcolor=\color{white}, 
	%\tiny < \scriptsize < \footnotesize < \small < \normalsize < \large < \Large < \LARGE < \huge < \Huge
	basicstyle = \footnotesize,       
	breakatwhitespace = false,        
	breaklines = true,                 
	captionpos = b,                    
	commentstyle = \color{mygreen}\bfseries,
	extendedchars = false,
	frame = shadowbox, 
	framerule=0.5pt,
	keepspaces=true,
	keywordstyle=\color{blue}\bfseries, % keyword style
	language = C++,                     % the language of code
	otherkeywords={string}, 
	numbers=left, 
	numbersep=5pt,
	numberstyle=\tiny\color{mygray},
	rulecolor=\color{black},         
	showspaces=false,  
	showstringspaces=false, 
	showtabs=false,    
	stepnumber=1,         
	stringstyle=\color{mymauve},        % string literal style
	tabsize=4,          
	title=\lstname           
}

%\sum\nolimits_{j=1}^{M}   上下标位于求和符号的水平右端,
%\sum\limits_{j=1}^{M}   上下标位于求和符号的上下处,
%\sum_{j=1}^{M}  对上下标位置没有设定,会随公式所处环境自动调整。

%%%%%%%%%%%%%画图包%%%%%%%%%%%%%
\usepackage{tikz}
%%%%%%%%%%%%%好看的矩形%%%%%%%%%%%%%
\tikzset{
	rect1/.style = {
		shape = rectangle,% 指定样式
		minimum height=2cm,% 最小高度
		minimum width=4cm,% 最小宽度
		align = center,% 文字居中
		drop shadow,% 阴影
	}
}
%%%%%%%%%%%%%画图背景包%%%%%%%%%%%%%
\usetikzlibrary{backgrounds}

%%%%%%%%%%%%%在tikz中画一个顶点%%%%%%%%%%%%%
%%%%%%%%%%%%%#1:node名称%%%%%%%%%%%%%
%%%%%%%%%%%%%#2:位置%%%%%%%%%%%%%
%%%%%%%%%%%%%#3:标签%%%%%%%%%%%%%
\newcommand{\newVertex}[3]{\node[circle, draw=black, line width=1pt, scale=0.8] (#1) at #2{#3}}
%%%%%%%%%%%%%在tikz中画一条边%%%%%%%%%%%%%
\newcommand{\newEdge}[2]{\draw [black,very thick](#1)--(#2)}
%%%%%%%%%%%%%在tikz中放一个标签%%%%%%%%%%%%%
%%%%%%%%%%%%%#1:名称%%%%%%%%%%%%%
%%%%%%%%%%%%%#2:位置%%%%%%%%%%%%%
%%%%%%%%%%%%%#3:标签内容%%%%%%%%%%%%%
\newcommand{\newLabel}[3]{\node[line width=1pt] (#1) at #2{#3}}

%%%%%%%%%%%%%强制跳过一行%%%%%%%%%%%%%
\newcommand{\jumpLine} {\hspace*{\fill} \par}
%%%%%%%%%%%%%关键点指令,可用itemise替代%%%%%%%%%%%%%
\newcommand{\keypoint}[2]{$\bullet$\textbf{#1}\quad#2\par}
%%%%%%%%%%%%%<T>平均值表示%%%%%%%%%%%%%
\newcommand{\average}[1]{\left\langle #1\right\rangle }
%%%%%%%%%%%%%表格内嵌套表格%%%%%%%%%%%%%
\newcommand{\tabincell}[2]{\begin{tabular}{@{}#1@{}}#2\end{tabular}}
%%%%%%%%%%%%%大黑点item头%%%%%%%%%%%%%
\newcommand{\itemblt}{\item[$\bullet$]}
%%%%%%%%%%%%%大圈item头%%%%%%%%%%%%%
\newcommand{\itemc}{\item[$\circ$]}
%%%%%%%%%%%%%大星星item头%%%%%%%%%%%%%
\newcommand{\itembs}{\item[$\bigstar$]}
%%%%%%%%%%%%%右▷item头%%%%%%%%%%%%%
\newcommand{\itemrhd}{\item[$\rhd$]}
%%%%%%%%%%%%%定义为%%%%%%%%%%%%%
\newcommand{\defas}{=_{df}}
%%%%%%%%%%%%%偏导%%%%%%%%%%%%%
\newcommand{\partialx}[2]{\frac{\partial #1}{\partial #2}}
%%%%%%%%%%%%%蕴含%%%%%%%%%%%%%
\newcommand{\imp}{\rightarrow}
%%%%%%%%%%%%%上取整%%%%%%%%%%%%%
\newcommand{\ceil}[1]{\lceil#1\rceil}
%%%%%%%%%%%%%下取整%%%%%%%%%%%%%
\newcommand{\floor}[1]{\lfloor#1\rfloor}

%%%%%%%%%%%%%双线分割线%%%%%%%%%%%%%
\newcommand*{\doublerule}{\hrule width \hsize height 1pt \kern 0.5mm \hrule width \hsize height 2pt}
%%%%%%%%%%%%%双线中间可加东西的分割线%%%%%%%%%%%%%
\newcommand\doublerulefill{\leavevmode\leaders\vbox{\hrule width .1pt\kern1pt\hrule}\hfill\kern0pt }
%%%%%%%%%%%%%左大括号%%%%%%%%%%%%%
\newcommand{\leftbig}[1]{\left\{\begin{array}{l}#1\end{array}\right.}
%%%%%%%%%%%%%矩阵%%%%%%%%%%%%%
\newcommand{\mat}[2]{\left[\begin{array}{#1}#2\end{array}\right]}
%%%%%%%%%%%%%组合%%%%%%%%%%%%%
\newcommand{\comb}[2]{\left(\begin{array}{c}#1 \\ #2\end{array}\right)}
%%%%%%%%%%%%%可换行圆角文本框%%%%%%%%%%%%%
\newcommand{\ovalboxn}[1]{\ovalbox{\tabincell{l}{#1}}}
%%%%%%%%%%%%%设置section的counter, 使从1开始%%%%%%%%%%%%%
\setcounter{section}{0}

%%%%%%%%%%%%%Colors%%%%%%%%%%%%%
\newcommand{\lightercolor}[3]{% Reference Color, Percentage, New Color Name
	\colorlet{#3}{#1!#2!white}
}
\newcommand{\darkercolor}[3]{% Reference Color, Percentage, New Color Name
	\colorlet{#3}{#1!#2!black}
}
\definecolor{aquamarine}{rgb}{0.5, 1.0, 0.83}
\definecolor{Seashell}{RGB}{255, 245, 238} %背景色浅一点的
\definecolor{Firebrick4}{RGB}{255, 0, 0}%文字颜色红一点的
\lightercolor{gray}{20}{lgray}
\newcommand{\hlg}[1]{
	\begingroup
	\sethlcolor{lgray}%背景色
	\textcolor{black}{\hl{\mbox{#1}}}%textcolor里面对应文字颜色
	\endgroup
}



\title{大数据算法 HW2}
\author{PB18111697 王章瀚}

\begin{document}
\maketitle
\section*{1.}

\noindent \textbf{我们对一个 $d$ 维欧氏空间的点集做 k-center 聚类. 假设 $d$ 和 $k$ 都为常数. 任给 $\epsilon > 0$, 我们是否能给出一个 $(1 + \epsilon)$ 倍近似比解? (35分)}

\section*{2.}

\noindent \textbf{对于 k-means 聚类, 如果 $k$ 为常数, 且我们假设在最优解中, 每一个 cluster 大小的下限为 $\alpha n$ ($n$ 为点的个数, $0 < \alpha < 1/k$), 我们能否通过简单的均匀采样得到一个具有常数近似比的初始解? (35分)} \\\jumpLine\noindent
考虑通过简单的均匀采样得到一个具有常数近似比的初始解的概率:
$$Pr\left[ \sum\limits_{i=1}^n\|p_i-C(p_i)\|^2 \le (1+\epsilon)\sum\limits_{i=1}^n\|p_i-C^{opt}(p_i)\|^2\right]$$
根据 Markov's 不等式:
$$Pr\left[ \sum\limits_{i=1}^n\|p_i-C(p_i)\|^2 \le (1+\epsilon)\sum\limits_{i=1}^n\|p_i-C^{opt}(p_i)\|^2\right] \ge 1 - \frac{E\left[\sum\limits_{i=1}^n\|p_i-C(p_i)\|^2 \right]}{(1+\epsilon)\sum\limits_{i=1}^n\|p_i-C^{opt}(p_i)\|^2}$$
下面计算 $E\left[\sum\limits_{i=1}^n\|p_i-C(p_i)\|^2 \right]$:
\begin{align*}
	E\left[\sum\limits_{i=1}^n\|p_i-C(p_i)\|^2 \right]
	&= \sum\limits_{i=1}^n E\left( p_i^2 + C^2(p_i) - 2\left< p_i, C(p_i) \right> \right)\\
	&= 
\end{align*}
参考文献 Fast k-Means Algorithms with Constant Approximation\url{https://link.springer.com/content/pdf/10.1007%2F11602613.pdf} 所给出的 Algorithm1, 可以做到常数近似比. 其主要思想是用抽样的结果来估计指定的 $k$ 个簇中心给出的目标函数值, 从而在减少计算量的前提下, 给出常数倍近似比.

\subsection*{算法内容}

\subsection*{近似比证明}

它的近似比

\section*{3.}

\noindent \textbf{gilbert's 算法的描述是基于欧氏空间. 如果数据经过某个 kernel function 映射到一个新的空间 $\Pi$(比如每个点 $p$ 被映射到 $\phi(p)\in\Pi$), 我们能否利用 gilbert's 算法在空间 $\Pi$ 中计算 polytope distance? (提示: 在空间 $\Pi$ 中, 我们可以通过 kerenel function K 得到任意两点的内积 $K(\phi(p), \phi(q))$) (30分)} \\\jumpLine\noindent
考虑映射后的点集, 记为 $\phi(P^+)$ 和 $\phi(P^-)$.
gilbert 算法中, 我们需要
\begin{enumerate}
	\item pick $q_0 \in Q$, 使之最接近原点(实际不一定要最近). 令 $x_1 = q_0$
	\item 重复以下步骤: 取 $q_i \in Q$, 满足 $proj_{\vec{x}_i}(q_i)$ 最小. 令 $x_{i+1}$ 为直线段 $\overline{q_i x_i}$ 上离原点最近的点. 
\end{enumerate}
稍作改进, 可以改为核形式:
\begin{enumerate}
	\item pick $q_0 \in Q$, 使 $\phi(q_0)$ 最接近原点(也就是 $K(\phi(q_0), \phi(q_0))$ 最小(实际不一定要最近). 令 $x_1 = q_0$
	\item 重复以下步骤: 取 $\phi(q_i) =\arg\min\limits_{q_i \in Q} proj_{\phi({x}_i)}(\phi(q_i))=\frac{K(\phi(q_i), \phi(x_i))}{\sqrt{K(\phi(x_i), `'\phi(x_i))}}$ 最小.  \\
	令 $\phi(x_{i+1})$ 为直线段 $\overline{\phi(q_i) \phi(x_i)}$ 上离原点最近的点. 
	\begin{itemize}
		\item 这里的最接近的点 $\phi(x_{i+1})0$可以这样给出: 由于直线段 $\overline{\phi(q_i) \phi(x_i)}$ 上的点可以表示为 $\phi(q_i), \phi(x_i)$ 的凸组合, 即 $x_{i+1}=\alpha\phi(q_i) + (1-\alpha) \phi(x_i)$. \\
		为使之最小, 可以最小化
		\begin{align*}
			&\qquad\|\alpha\phi(q_i) + (1-\alpha) \phi(x_i)\|^2 \\
			&=\alpha^2 K^2(\phi(x_i), \phi(x_i)) + (1-\alpha)^2 K^2(\phi(q_i), \phi(q_i)) + 2\alpha(1-\alpha)K(\phi(x_i), \phi(q_i))
		\end{align*}
		这仅仅是一个关于 $\alpha$ 的二次方程, 取最值时有
		$$\alpha=\frac{K(\phi(x_i), \phi(q_i)) - K^2(\phi(q_i), \phi(q_i))}{K^2(\phi(x_i), \phi(x_i)) + K^2(\phi(q_i), \phi(q_i)) - 2K(\phi(x_i), \phi(q_i))}$$
	\end{itemize}
\end{enumerate}
迭代足够多次后, polytope distance 在 $\Pi$ 空间就由 $\|\phi(x_i)\|_2^2$ 给出, 通过核函数和两点内积的关系就可以推得原空间的 polytope distance.

\end{document}



