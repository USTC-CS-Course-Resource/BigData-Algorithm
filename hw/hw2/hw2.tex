\documentclass[UTF8]{article}
\usepackage{graphicx}
\usepackage{subfigure}
\usepackage{amsmath}
\usepackage{makecell}
\usepackage[utf8]{inputenc}
\usepackage[space]{ctex} %中文包
\usepackage{listings} %放代码
\usepackage{xcolor} %代码着色宏包
\usepackage{CJK} %显示中文宏包
\usepackage{float}
\usepackage{diagbox}
\usepackage{bm}
\usepackage{ulem} 
\usepackage{amssymb}
\usepackage{url}
\usepackage{soul}
\usepackage{color}
\usepackage{geometry}
\usepackage{fancybox} %花里胡哨的盒子
\usepackage{xhfill} %填充包, 可画分割线 https://www.latexstudio.net/archives/8245
\usepackage{multicol} %多栏包
\usepackage{enumerate} %可以方便地自定义枚举标题
\usepackage{enumitem}
\usepackage{multirow} %表格中多行单元格合并
\usepackage{wasysym} %可以使用wasysym里的一堆奇奇怪怪的符号
\usepackage{hyperref} % url
%%%%%%%%%%%%%%%伪代码%%%%%%%%%%%%%%%
\usepackage{amsmath}
\usepackage{algorithm}
\usepackage{algorithmicx}
\usepackage[noend]{algpseudocode}
%%%%%%%%%%%%%%%画图包%%%%%%%%%%%%%%%
\usepackage{tikz}
\usepackage{pgfplots} % http://pgfplots.sourceforge.net/gallery.html
\usetikzlibrary{pgfplots.patchplots} % 拟合支持
\usetikzlibrary{arrows,shapes,automata,petri,positioning,calc} % 状态图支持
\usetikzlibrary{arrows.meta} % 箭头
\usetikzlibrary{shadows} % 阴影支持
\usepackage{forest} % 画树

\geometry{left = 1.5cm, right = 1.5cm, top=1.5cm, bottom=2cm}

\definecolor{mygreen}{rgb}{0,0.6,0}
\definecolor{mygray}{rgb}{0.5,0.5,0.5}
\definecolor{mymauve}{rgb}{0.58,0,0.82}
\lstset{
	backgroundcolor=\color{white}, 
	%\tiny < \scriptsize < \footnotesize < \small < \normalsize < \large < \Large < \LARGE < \huge < \Huge
	basicstyle = \footnotesize,       
	breakatwhitespace = false,        
	breaklines = true,                 
	captionpos = b,                    
	commentstyle = \color{mygreen}\bfseries,
	extendedchars = false,
	frame = shadowbox, 
	framerule=0.5pt,
	keepspaces=true,
	keywordstyle=\color{blue}\bfseries, % keyword style
	language = C++,                     % the language of code
	otherkeywords={string}, 
	numbers=left, 
	numbersep=5pt,
	numberstyle=\tiny\color{mygray},
	rulecolor=\color{black},         
	showspaces=false,  
	showstringspaces=false, 
	showtabs=false,    
	stepnumber=1,         
	stringstyle=\color{mymauve},        % string literal style
	tabsize=4,          
	title=\lstname           
}

%\sum\nolimits_{j=1}^{M}   上下标位于求和符号的水平右端,
%\sum\limits_{j=1}^{M}   上下标位于求和符号的上下处,
%\sum_{j=1}^{M}  对上下标位置没有设定,会随公式所处环境自动调整。

%%%%%%%%%%%%%画图包%%%%%%%%%%%%%
\usepackage{tikz}
%%%%%%%%%%%%%好看的矩形%%%%%%%%%%%%%
\tikzset{
	rect1/.style = {
		shape = rectangle,% 指定样式
		minimum height=2cm,% 最小高度
		minimum width=4cm,% 最小宽度
		align = center,% 文字居中
		drop shadow,% 阴影
	}
}
%%%%%%%%%%%%%画图背景包%%%%%%%%%%%%%
\usetikzlibrary{backgrounds}

%%%%%%%%%%%%%在tikz中画一个顶点%%%%%%%%%%%%%
%%%%%%%%%%%%%#1:node名称%%%%%%%%%%%%%
%%%%%%%%%%%%%#2:位置%%%%%%%%%%%%%
%%%%%%%%%%%%%#3:标签%%%%%%%%%%%%%
\newcommand{\newVertex}[3]{\node[circle, draw=black, line width=1pt, scale=0.8] (#1) at #2{#3}}
%%%%%%%%%%%%%在tikz中画一条边%%%%%%%%%%%%%
\newcommand{\newEdge}[2]{\draw [black,very thick](#1)--(#2)}
%%%%%%%%%%%%%在tikz中放一个标签%%%%%%%%%%%%%
%%%%%%%%%%%%%#1:名称%%%%%%%%%%%%%
%%%%%%%%%%%%%#2:位置%%%%%%%%%%%%%
%%%%%%%%%%%%%#3:标签内容%%%%%%%%%%%%%
\newcommand{\newLabel}[3]{\node[line width=1pt] (#1) at #2{#3}}

%%%%%%%%%%%%%强制跳过一行%%%%%%%%%%%%%
\newcommand{\jumpLine} {\hspace*{\fill} \par}
%%%%%%%%%%%%%关键点指令,可用itemise替代%%%%%%%%%%%%%
\newcommand{\keypoint}[2]{$\bullet$\textbf{#1}\quad#2\par}
%%%%%%%%%%%%%<T>平均值表示%%%%%%%%%%%%%
\newcommand{\average}[1]{\left\langle #1\right\rangle }
%%%%%%%%%%%%%表格内嵌套表格%%%%%%%%%%%%%
\newcommand{\tabincell}[2]{\begin{tabular}{@{}#1@{}}#2\end{tabular}}
%%%%%%%%%%%%%大黑点item头%%%%%%%%%%%%%
\newcommand{\itemblt}{\item[$\bullet$]}
%%%%%%%%%%%%%大圈item头%%%%%%%%%%%%%
\newcommand{\itemc}{\item[$\circ$]}
%%%%%%%%%%%%%大星星item头%%%%%%%%%%%%%
\newcommand{\itembs}{\item[$\bigstar$]}
%%%%%%%%%%%%%右▷item头%%%%%%%%%%%%%
\newcommand{\itemrhd}{\item[$\rhd$]}
%%%%%%%%%%%%%定义为%%%%%%%%%%%%%
\newcommand{\defas}{=_{df}}
%%%%%%%%%%%%%偏导%%%%%%%%%%%%%
\newcommand{\partialx}[2]{\frac{\partial #1}{\partial #2}}
%%%%%%%%%%%%%蕴含%%%%%%%%%%%%%
\newcommand{\imp}{\rightarrow}
%%%%%%%%%%%%%上取整%%%%%%%%%%%%%
\newcommand{\ceil}[1]{\lceil#1\rceil}
%%%%%%%%%%%%%下取整%%%%%%%%%%%%%
\newcommand{\floor}[1]{\lfloor#1\rfloor}

%%%%%%%%%%%%%双线分割线%%%%%%%%%%%%%
\newcommand*{\doublerule}{\hrule width \hsize height 1pt \kern 0.5mm \hrule width \hsize height 2pt}
%%%%%%%%%%%%%双线中间可加东西的分割线%%%%%%%%%%%%%
\newcommand\doublerulefill{\leavevmode\leaders\vbox{\hrule width .1pt\kern1pt\hrule}\hfill\kern0pt }
%%%%%%%%%%%%%左大括号%%%%%%%%%%%%%
\newcommand{\leftbig}[1]{\left\{\begin{array}{l}#1\end{array}\right.}
%%%%%%%%%%%%%矩阵%%%%%%%%%%%%%
\newcommand{\mat}[2]{\left[\begin{array}{#1}#2\end{array}\right]}
%%%%%%%%%%%%%组合%%%%%%%%%%%%%
\newcommand{\comb}[2]{\left(\begin{array}{c}#1 \\ #2\end{array}\right)}
%%%%%%%%%%%%%可换行圆角文本框%%%%%%%%%%%%%
\newcommand{\ovalboxn}[1]{\ovalbox{\tabincell{l}{#1}}}
%%%%%%%%%%%%%设置section的counter, 使从1开始%%%%%%%%%%%%%
\setcounter{section}{0}

%%%%%%%%%%%%%Colors%%%%%%%%%%%%%
\newcommand{\lightercolor}[3]{% Reference Color, Percentage, New Color Name
	\colorlet{#3}{#1!#2!white}
}
\newcommand{\darkercolor}[3]{% Reference Color, Percentage, New Color Name
	\colorlet{#3}{#1!#2!black}
}
\definecolor{aquamarine}{rgb}{0.5, 1.0, 0.83}
\definecolor{Seashell}{RGB}{255, 245, 238} %背景色浅一点的
\definecolor{Firebrick4}{RGB}{255, 0, 0}%文字颜色红一点的
\lightercolor{gray}{20}{lgray}
\newcommand{\hlg}[1]{
	\begingroup
	\sethlcolor{lgray}%背景色
	\textcolor{black}{\hl{\mbox{#1}}}%textcolor里面对应文字颜色
	\endgroup
}



\title{大数据算法 HW2}
\author{PB18111697 王章瀚}

\begin{document}
\maketitle
\section*{1.}

\noindent \textbf{我们对一个 $d$ 维欧氏空间的点集做 k-center 聚类. 假设 $d$ 和 $k$ 都为常数. 任给 $\epsilon > 0$, 我们是否能给出一个 $(1 + \epsilon)$ 倍近似比解? (35分)} \\\jumpLine\noindent
%查找资料发现, Dorit S. Hochbaum and David B. Shmoys 在 \href{https://www.jstor.org/stable/3689371?seq=1\#metadata_info_tab_contents}{\textbf{\textit{A Best Possible Heuristic for the k-Center Problem}}}\footnote{\textbf{\textit{A Best Possible Heuristic for the k-Center Problem}}: https://www.jstor.org/stable/3689371?seq=1\#metadata\_info\_tab\_contents}
%中证明了任何近似比小于 2 的 k-center 算法都蕴含 $P=NP$. 因此若我们能找到满足题意的多项式时间的算法, 就相当于证明了 $P=NP$. 这已超出我的能力范围. 
\noindent
本题参考 \href{http://graphics.stanford.edu/courses/cs468-06-winter/Papers/BHI02.pdf}{\textbf{\textit{Approximate Clustering via Core-Sets}}}\footnote{\textbf{\textit{Approximate Clustering via Core-Sets}}: http://graphics.stanford.edu/courses/cs468-06-winter/Papers/BHI02.pdf}
中提到的算法完成.

\subsection*{算法说明}
\noindent
\begin{algorithm}[H]
	\caption{
		$(1+\epsilon)$-approximation for k-center
	}  
	\begin{algorithmic}[1] %每行显示行号
%		\Function{Coreset-$k$-centers}{${P}:$ point set; $k:$ \#centers, $\epsilon$}
%		\State $S_i\leftarrow \varnothing, i=1\ldots k$
%		\State
%		\Return $\{c_i|i=1,\ldots,k\}$ and radius $r = (1+\epsilon)\operatorname{max}(r_1,\ldots,r_k)$
%		\EndFunction
%		\State
		\Function{Sub-$k$-centers}{$\{S_i\}:$ current coreset; ${P}:$ Remaining point set; $k:$ \#centers}
		\For{$i=1$ to $O(1/\epsilon^2)$}
		\For{$j=1$ to $k$}
		\State Let $B_j(c_j, r_j) =\operatorname{MEB}(S_j)$
		\EndFor
		\State $
		p=\operatorname{argmax}_{p \in P}\left(\min \left(\left\|p-c_{j}\right\|, j=1,\ldots k\right)\right.
		$
		\State \Return the best solution of \textsc{Sub-$k$-centers}($\{S_1,\ldots,S_{j-1},\{p\}\cup S_j,S_{j+1},\ldots,S_k\}$, $P/\{p\}$, $k$), $\forall j = 1,\cdots, k$
		\EndFor
		\EndFunction
	\end{algorithmic}
\end{algorithm}
\subsection*{近似比证明}
\noindent 其证明的核心是下面这个引理:
\subsubsection*{Lemma 1}
\noindent \textbf{存在一个子集 $S \subseteq P$, $|S|=O(1/\epsilon^2)$, 使得 S 的 MEB 的半径至少是 $\frac{1}{1+\epsilon}$ 倍 P 的 MEB 半径.} \\\jumpLine
\noindent
其该证明可以在论文的 Lemma2.3 中找到.
\subsubsection*{近似比}
\noindent 有了 Lemma 1, 就能够保证上述算法返回的半径是 $(1+\epsilon)r$ 的. 但具体细节我其实没太弄懂, 并且该论文只证明了 1-center 和 2-center 的情况, 对于 k-center 的情况他表示容易扩展, 但我没有想明白怎么扩展, 就不在这里伪证了.

\subsection*{另一种方法}
\noindent \textit{这是和同学讨论的一个方法, 有必要记录一下.} \\
\noindent \textbf{算法步骤}如下:
\begin{enumerate}
	\item 先对原有数据集 $P$ 做冈萨雷斯算法, 得到一个近似比为 2 的结果 $r^*$, 
	\item 然后考虑在包围所有点的一个有限范围内, 将 $d$ 维空间划分为边长为 $\frac{\epsilon r^*/2}{n\sqrt{d}}$ 的超立方体. 
	\item 遍历超立方体的每个顶点构成集合的 k-子集, 找出满足目标函数的最有情况, 并作为最优解返回即可.
\end{enumerate}


\subsubsection*{近似比证明}
\noindent 每个超立方体边长是 $\delta$.
那么考虑点 $p$ 的簇中心是 $c$, 则对任意 $c$ 所在超立方体的顶点 $q$, 有
$$\|p-q\| \le \|p-c\| + \delta\sqrt{d}$$
所以, 对所有点来说, 有
$$\max(\|p-q\|) \le \max(\|p-c\| + n\delta\sqrt{d}) \le \max\|p-c\| + n\delta\sqrt{d} \le r_{opt} + n\delta\sqrt{d}$$
要想满足 $(1+\epsilon)$ 的近似比, 只要 $n\delta\sqrt{d}\le \epsilon r_{opt}$ 即可.
虽然我们无法得知 $r_{opt}$ 是多少, 但我们可以用冈萨雷斯算法得到一个不超过 $2r_{opt}$ 的结果, 将这个结果的半径除以二就能得到一个足够小的半径, 以估计 $\delta$.\\
假设冈萨雷斯算法给出 $r^*$, 那么只要 $\delta \le \frac{\epsilon r^*/2}{n\sqrt{d}}$ 即可满足 $(1+\epsilon)$ 的近似比.




\section*{2.}

\noindent \textbf{对于 k-means 聚类, 如果 $k$ 为常数, 且我们假设在最优解中, 每一个 cluster 大小的下限为 $\alpha n$ ($n$ 为点的个数, $0 < \alpha < 1/k$), 我们能否通过简单的均匀采样得到一个具有常数近似比的初始解? (35分)} \\\jumpLine\noindent
本题参考 \href{https://link.springer.com/content/pdf/10.1007\%2F11602613.pdf}{\textbf{\textit{Fast k-Means Algorithms with Constant Approximation}}}\footnote{\textbf{\textit{Fast k-Means Algorithms with Constant Approximation}}: https://link.springer.com/content/pdf/10.1007\%2F11602613.pdf}
中的 Algorithm1 完成.其主要思想是用抽样的结果来估计指定的 $k$ 个簇中心给出的目标函数值, 从而在减少计算量的前提下, 给出常数倍近似比. 

\subsection*{算法内容}
考虑输入原始数据点集为 $P$, 若每个 cluster 的大小下限为 $\alpha n$, 则只要从 $P$ 中做大小为 $\frac{8}{\epsilon\alpha}$ 的随机采样 $T$. 遍历 $T$ 上 所有 k 大小的点集作为簇中心, 并计算目标函数. 而后将目标函数值最小的 k 大小点集作为最终结果. \\
这样, 可以至少有 $\frac{1}{12} - \exp(1-\frac{1}{\epsilon})$ 的概率得到 $(5+2\epsilon)$ 的常数近似比.

\subsection*{算法性质证明}
\subsubsection*{Lemma 1}
\noindent \textbf{若 $T$ 是原始数据 $P$ 的一个随机抽样, 大小为 $|T|$. 而 $\mu_P$ 是 $P$ 的中心, $\mu_T$ 是 $T$ 的中心, 那么有至少 $1-\delta$ $(\delta > 0)$ 的概率使下式成立:
$$\sum\limits_{x_i \in P}\|x_i-\mu_T\|^2 \le (1+\frac{1}{\delta|T|})\sum\limits_{x_i\in P}\|x_i-\mu_P\|^2$$}
\noindent
\textbf{证明}: 对于不等式左边有:
\begin{align*}
	\sum\limits_{x_i \in P}\|x_i-\mu_T\|^2
	&=\sum\limits_{x_i \in P} \|x_i-\mu_P + \mu_P -\mu_T\|^2 \\
	&=\sum\limits_{x_i \in P} \|x_i-\mu_P\|^2 + |P|\|\mu_P -\mu_T\|^2
\end{align*}
考虑其第二项, 由 Markov 不等式有:
$$Pr\left[\|\mu_P -\mu_T\|^2 \ge \frac{1}{\delta |P|}E[\|\mu_P -\mu_T\|^2]\right] \le \delta$$
因此有至少 $(1-\delta)$ 的概率能够保证:
\begin{align*}
	\sum\limits_{x_i \in P}\|x_i-\mu_T\|^2
	&=\sum\limits_{x_i \in P} \|x_i-\mu_P + \mu_P -\mu_T\|^2 \\
	&=\sum\limits_{x_i \in P} \|x_i-\mu_P\|^2 + |P|\|\mu_P -\mu_T\|^2 \\
	&\le \sum\limits_{x_i \in P} \|x_i-\mu_P\|^2 + \frac{1}{\delta}\|\mu_P -\mu_T\|^2 \\
	&= \sum\limits_{x_i \in P} \|x_i-\mu_P\|^2 + \frac{1}{\delta |T|}|T|\|\mu_P -\mu_T\|^2 \\
	&\le \sum\limits_{x_i \in P} \|x_i-\mu_P\|^2 + \frac{1}{\delta |T|}\|\mu_P -x_i\|^2 \\
\end{align*}

\subsubsection*{Lemma 2}
\noindent \textbf{令 $C_T$ 表示样本中离样本中心最近的点, 那么可以以至少 $\frac{1}{12}$ 的概率保证
$$\sum\limits_{x_i\in P}\|x_i-C_T\|^2 \le (5+2\epsilon)\sum\limits_{x_i\in P}\|x_i-\mu_P\|^2$$}
\noindent
\textbf{证明}: 考虑三角不等式
$$\|x_i-C_T\|^2 \le 2(\|x_i-\mu_T\|^2 + \|C_T-\mu_T\|^2)$$
对右式第二项求和有
$$\sum\limits_{x_i \in P}\|C_T-\mu_T\|^2 = |P|\|C_T-\mu_T\|^2 \le \frac{|P|}{|T|}\sum\limits_{x_i \in P}\sum\limits_{x_i \in P}\|x_i - \mu_T\|^2 = |P|Var(T)$$
这里 $Var(T)=\frac{1}{|T|}\sum_{x_i \in P}\|x_i-\mu_T\|^2$. 因为 $T$ 是原始数据 $P$ 的抽样, 根据概率论基础知识有:
$$E(Var(T))=\frac{|T|-1}{|T|}Var(P)$$
再由 Markov 不等式可以得到:
$$Pr[Var(T) \le 1.5Var(P)] \ge 1-\frac{|T|-1}{1.5|T|} > \frac{1}{3}$$
因此, 下式成立的概率至少是 $\frac{1}{3}$:
$$\sum\limits_{x_i\in P}\|C_T-\mu_T\|^2 \le 1.5|P|Var(P) = 1.5\sum\limits_{x_i \in P}\|x_i - \mu_P\|^2$$
记该事件成立为事件 A, 而 Lemma 1 的描述的内容(令 $\delta=\frac{1}{4}$) 为事件 B,那么有:
$$Pr[AB]=1-Pr(\bar{A}\cup\bar{B})\ge1-(Pr(\bar{A}+Pr(\bar{B})))=Pr(A)+Pr(B)-1 > \frac{3}{4} + \frac{1}{3} - 1 = \frac{1}{12}$$

\subsubsection*{近似比证明}
\noindent 为了满足上面的不等式, 我们应当能够抽样 $T$ 使得它能包含每个 cluster 的至少 $\frac{4}{\epsilon}$ 个点. \\
假设 $n_s$ 是最小 cluster $S$ 的大小, 由题设知, $n_s=\alpha|P|=\alpha k\frac{|P|}{k}$. \\
设 $X_s$ 是 $T$ 中落于最小 cluster $S$ 的点. 根据 Chernoff Bound 的乘积形式, 我们有:
$$Pr\left[X_s \ge \beta\left(\frac{|T|}{|P|}n_s\right)\right] \ge 1 - \exp\left(-\frac{(1-\beta)^2}{2}\left(\frac{|T|}{|P|}n_s\right)\right)$$
也就是
$$Pr\left[X_s \ge \beta|T|\alpha\right] \ge 1 - \exp\left(-\frac{(1-\beta)^2}{2}|T|\alpha\right)$$
因此只需要让 $\beta|T|\alpha=\frac{4}{\epsilon}$ 且 $\beta=\frac{1}{2}$, 也就是 $|T|=\frac{8}{\epsilon\alpha}$, 则样本数量满足要求的概率至少有 $1-\exp(1-\frac{1}{\epsilon})$\\
\noindent
综上所述, 至少有 $\frac{1}{12} - \exp(1-\frac{1}{\epsilon})$ 的概率能够满足 $(5+2\epsilon)$ 的近似比, 即:
$$\sum\limits_{x_i\in P}\|x_i-C_T\|^2 \le (5+2\epsilon)\sum\limits_{x_i\in P}\|x_i-\mu_P\|^2$$

\section*{3.}
\noindent \textbf{gilbert's 算法的描述是基于欧氏空间. 如果数据经过某个 kernel function 映射到一个新的空间 $\Pi$(比如每个点 $p$ 被映射到 $\phi(p)\in\Pi$), 我们能否利用 gilbert's 算法在空间 $\Pi$ 中计算 polytope distance? (提示: 在空间 $\Pi$ 中, 我们可以通过 kerenel function K 得到任意两点的内积 $K(\phi(p), \phi(q))$) (30分)} \\\jumpLine\noindent
在原本的 gilbert 算法中, 我们需要
\begin{enumerate}
	\item pick $q_0 \in Q$, 使之最接近原点(实际不一定要最近). 令 $x_1 = q_0$
	\item 重复以下步骤: 取 $q_i \in Q$, 满足 $proj_{\vec{x}_i}(q_i)$ 最小. 令 $x_{i+1}$ 为直线段 $\overline{q_i x_i}$ 上离原点最近的点. 
\end{enumerate}
考虑映射后的点集, 记为 $\phi(P^+)$ 和 $\phi(P^-)$. 将上述 Gilbert 算法稍作改进, 可以改为核形式:
\begin{enumerate}
	\item pick $q_0 \in Q$, 使 $\phi(q_0)$ 最接近原点(也就是 $K(\phi(q_0), \phi(q_0))$ 最小(实际不一定要最近). 令 $x_1 = q_0$
	\item 重复以下步骤: 取 $\phi(q_i) =\arg\min\limits_{q_i \in Q} proj_{\phi({x}_i)}(\phi(q_i))=\frac{K(\phi(q_i), \phi(x_i))}{\sqrt{K(\phi(x_i), `'\phi(x_i))}}$ 最小.  \\
	令 $\phi(x_{i+1})$ 为直线段 $\overline{\phi(q_i) \phi(x_i)}$ 上离原点最近的点. 
	\begin{itemize}
		\item 这里的最接近的点 $\phi(x_{i+1})0$可以这样给出: 由于直线段 $\overline{\phi(q_i) \phi(x_i)}$ 上的点可以表示为 $\phi(q_i), \phi(x_i)$ 的凸组合, 即 $x_{i+1}=\alpha\phi(q_i) + (1-\alpha) \phi(x_i)$. \\
		为使之最小, 可以最小化
		\begin{align*}
			&\qquad\|\alpha\phi(q_i) + (1-\alpha) \phi(x_i)\|^2 \\
			&=\alpha^2 K(\phi(x_i), \phi(x_i)) + (1-\alpha)^2 K(\phi(q_i), \phi(q_i)) + 2\alpha(1-\alpha)K(\phi(x_i), \phi(q_i))
		\end{align*}
		这仅仅是一个关于 $\alpha$ 的二次方程, 取最值时有
		$$\alpha=\frac{K(\phi(x_i), \phi(q_i)) - K(\phi(q_i), \phi(q_i))}{K(\phi(x_i), \phi(x_i)) + K(\phi(q_i), \phi(q_i)) - 2K(\phi(x_i), \phi(q_i))}$$
	\end{itemize}
\end{enumerate}
迭代足够多次后, polytope distance 在 $\Pi$ 空间就可以直接由 $\|\phi(x_i)\|_2^2$ 给出.

\end{document}



